% first example chapter
% @author Andre Alves
%
\chapter{Introduction}\label{ch:introduction}
As a CJ1 project, this team chose to build a webapp that would allow a user to draw a circuit graphically and use that circuit to obtain \LaTeX code to draw a circuit using the \href{https://github.com/circuitikz/circuitikz}{CiruiTikZ} package.
The team developed a mostly functional MVP, and its source code is publicly available on \href{https://github.com/andre-a-alves/drawio-circuitikz}{GitHub}~\cite{sourcecode}.
Instead of building the webapp from scratch, the team forked the \href{https://github.com/jgraph/drawio}{draw.io} software by JGraph Ltd~\cite{drawioCode}, which was renamed \emph{diagrams.net} while working on this project.
Starting from that source code was essentially necessary due to the scope of the project, although that came with many of its own challenges that are captured in this document.

\section{Technology}\label{sec:technology}
This project employed a combination of Javascript and the \LaTeX package CircuiTikz~\cite{circuitikz}.
The Javascript project itself was built using Apache Ant, as was already implemented by diagrams.net.
Ant takes a source structure and transpiles the code into minimized javascript files that are a prepared webapp.
Due to its capabilities, IntelliJ IDEA Ultimate is the recommended IDE for this project.

\subsection{Diagrams.net}\label{subsec:diagrams-net}
Diagrams.net is an open-source, but closed-contribution, Javascript-based webapp that is used to draw diagrams, flowcharts, and similar visual documents.
It also includes support for drawing electrical circuits.
While diagrams.net offers support for drawing electrical circuits, the resulting circuits can only be exported as images.
Consequently, if included in a \LaTeX document, such as a thesis or a journal article, the resulting circuit does not match the overall style of the paper.

\subsection{CircuiTikZ}\label{subsec:circuitikz}
CircuiTikZ is an open-source \LaTeX package that is used for drawing circuits within \LaTeX documents.
It is a superset of the TikZ package.
As a text-based diagramming tool, creating CircuiTikZ diagrams is very labor-intensive, and the learning curve turns many people away from using it.
However, its text-based design results in a tremendous amount of flexibility, and the resulting circuits match the style of the paper they are made for.
Due to its incredible flexibility, the goal of this project was never to build a complete GUI for CircuiTikZ.
Instead, the goal of this project was to get a basic CircuiTikZ diagram that could then be further enhanced by hand, since the most difficult part of making a CircuiTikZ diagram is laying out the components.

\subsection{IntelliJ IDEA Ultimate}\label{subsec:intellij}
Jetbrains' flagship IDE IntelliJ IDEA Ultimate is a full-featured IDE that includes support for a number of languages and frameworks~\cite{intellij}.
In fact, it supports every framework and tool that is included in the original diagrams.net project, including Apache Ant.
When combined with the size and scope of the pre-existing project, this became the obvious tool for the job.\\

Although there is a free community edition of IntelliJ, the community edition lacks several of the features required by this project.
Luckily, students may receive a free education license of the tool as long as it is not used for commercial purposes.


\section{Completed Tasks Against Diagrams.net}\label{sec:completed-tasks}
The following is a list of tasks that were carried out directly against the diagrams.net project.
In retrospect, some of these tasks were unnecessary and were built into the application, already, but without public documentation, this was not known until they had already been performed.
Additionally, the number of necessary changes was such that this project will remain incompatible with the original project, anyway.
\begin{itemize}
 \item Remove non-electronics components, as well as components that are not a part of CircuiTikz.
 \item Reduce the number of options from the \texttt{File -> New} menu.
 \item Limit available themes to those that are based around components, only.
 \item Designed a \LaTeX export dialog.
 \item Added \LaTeX as an export option
\end{itemize}

\section{Additional Complete Tasks}\label{sec:additional-completed-tasks}
In addition to tasks that were performed directly against the diagrams.net project, the following is a general list of tasks that were performed independently.
Details of the parser structure are included separately in Chapter~\ref{ch:parser-structure}.
\begin{itemize}
 \item Classes were created to represent electrical components in an object-oriented way.
 \item An XML parser was built to parse the diagram's XML into those classes.
 \item All electrical components from both diagrams.net and CircuiTikz were cataloged and compared.
 \item A look up table was built to look up the above electrical components.
\end{itemize}

\section{Tasks Remaining}\label{sec:tasks-remaining}
While not all-inclusive, this is a short list of tasks that we recognize as remaining unsolved.
Further details for the parser itself can be found in Chapter~\ref{ch:parser-structure}.
\begin{itemize}
 \item Support needs to be added for CircuiTikz node-style components.
 \item Large sections of spaghetti code need to be refactored.
 \item Some components may still need to be removed from the webapp.
 \item The images of some removed components still need to be removed from the webapp.
\end{itemize}